% ===========================================
% HOMEWORK 3: Dice Roll
% Written by: Braidan Duffy
%
% Date: 12/28/2022
% Last Revision: 12/28/2022
% ============================================

\documentclass[
	letterpaper, % Page size
	fontsize=10pt, % Base font size
	twoside=true, % Use different layouts for even and odd pages (in particular, if twoside=true, the margin column will be always on the outside)
	%open=any, % If twoside=true, uncomment this to force new chapters to start on any page, not only on right (odd) pages
	%chapterentrydots=true, % Uncomment to output dots from the chapter name to the page number in the table of contents
	numbers=noenddot, % Comment to output dots after chapter numbers; the most common values for this option are: enddot, noenddot and auto (see the KOMAScript documentation for an in-depth explanation)
]{kaobook}

\begin{document}

\chapter*{Homework 3: Dice Roll}

\section*{Overview}
This homework assignment is the first assignment designed to get you familiar with the programming basics.
Your objective is to perform a series of simulated "dice rolls", perform a basic statistical analysis, and plot the results.
You can do this either in Python or MATLAB, but keep in mind that you should use whatever programming language your group is using for the research topic. \marginnote{\emph{For extra credit:} You may do this assignment in both languages}

You will first roll an individual dice 10 and 1000 times, calculating the mean, standard deviation, and histogram for each set.
Then, you will roll two dice, adding together their sums, 10 and 1000 times.
Again, calculating the mean, standard deviation, and histogram for each set of rolls.

\section*{Requirements}
To fulfill this project, you must successfully program a script that utilizes several different functions and loops to simulate dice rolls and calculate the statistics.
Each function should have a practical name, be easy to read and understand, and function.
You should also comment appropriately throughout the program such that someone else could understand what is occurring.

\section*{Submission}
You will submit the following to the Canvas assignment box:
\begin{enumerate}
    \item A summary document with tables of the statistical parameters and copies of the generated histograms
    \item A well-documented code file OR
    \item An organized zipped archive of well-documented code source files. Make sure the main script is named properly.
\end{enumerate}

\marginnote{The main script should be named in the following format: \lstinline{h[#]_[ASSIGNMENT NAME]_[LAST NAME].[EXTENSION]}, for example, \lstinline{h3_dice_roll_duffy.ipynb}}

\marginnote[-0.75in]{Documents that contain the figures and printouts with the code, such as Jupyter notebooks, will also be accepted in lieu of the other submission items.}

Note that any submitted code needs to be able to be executed without any errors. Any additional libraries need to be able to be installed using a package manager like pip.

\section*{Grading}
You will be graded on the following criteria:

\begin{margintable}
    \begin{tabular}{l | c}
        \toprule
        \multicolumn{1}{c|}{\textbf{Criterion}} & \textbf{Points} \\
        \midrule
        Efficacy & 60 \\
        Code Neatness & 20 \\
        Output Neatness & 20 \\
        Multilingual (EC) & 20 \\
        Additional Functions (EC) & 10 each \\
        \bottomrule
    \end{tabular}
\end{margintable}

\pagelayout{wide} % Remove margins

\section*{Extra Credit}
If you are willing to dig in a little bit more, this project has a couple of opportunities to earn extra credit points at the discretion of the instructor.
First, you may do this assignment in both Python and MATLAB. 
Being able to work with multiple programming languages is an extremely useful skill.
Python and MATLAB are also close enough in syntax that the code should not have to change much between them.

Additionally, you may write up to three different operations with the dice rolling and provide the same output for each.
For 10 points of extra credit each, you may write functions that subtract, multiply, and divide the rolls of two dice for 10 and 1000 rolls.
You will only be granted the extra credit points if the full output for each operation is provided!

\section*{Psuedocode}
For those that are new to programming, a suggested code layout is given below in psuedocode to guide you.

\begin{lstlisting}[style=kaolstplain, linewidth=1.5\textwidth]    
    Function: Roll One Die
        Argument: iterations
        Initialize zeroes array of length iterations
        For every iteration
            Sample a random integer from [1, 6]
            Add sample to array
        Returns: array of random samples, mean of array, stdev of array
    
    Function: Roll Two Dice and Sum
        Argument: iterations
        Initialize zeroes array of length iterations
        For every iteration
            Sample two random integers from [1, 6] and add together
            Add the summation to array
        Returns: array of summed samples, mean of array, stdev of array

    Calculate array, mean, and stdev for 10 single rolls
    Calculate array, mean, and stdev for 1000 single rolls

    Generate output table for single rolls and print

    Create singular die roll figure with two subplots
    Plot bar graph for 10 die rolls
    Plot bar graph for 1000 die rolls
    Show figure

    Calculate array, mean, stdev for 10 summed dice rolls
    Calculate array, mean, stdev for 1000 summed dice rolls

    Generate output table for summed rolls and print

    Create summed dice roll figure with two subplots
    Plot bar graph for 10 summed dice rolls
    Plot bar graph for 1000 summed dice rolls
    Show figure
    
    \end{lstlisting}

\end{document}