% ===========================================
% HOMEWORK 3: Dice Roll
% Written by: Braidan Duffy
%
% Date: 12/28/2022
% Last Revision: 12/28/2022
% ============================================

\documentclass[
	letterpaper, % Page size
	fontsize=10pt, % Base font size
	twoside=true, % Use different layouts for even and odd pages (in particular, if twoside=true, the margin column will be always on the outside)
	%open=any, % If twoside=true, uncomment this to force new chapters to start on any page, not only on right (odd) pages
	%chapterentrydots=true, % Uncomment to output dots from the chapter name to the page number in the table of contents
	numbers=noenddot, % Comment to output dots after chapter numbers; the most common values for this option are: enddot, noenddot and auto (see the KOMAScript documentation for an in-depth explanation)
]{kaobook}

\begin{document}

\chapter*{Homework 3: Dice Roll}

\section*{Overview}
This homework assignment is the first assignment designed to get you familiar with the programming basics.
Your objective is to perform a series of simulated "dice rolls", perform a basic statistical analysis, and plot the results.
You can do this either in Python or MATLAB, but keep in mind that you should use whatever programming language your group is using for the research topic. \marginnote{\emph{For extra credit:} You may do this assignment in both languages}

You will first roll an individual dice 10 and 1000 times, calculating the mean, standard deviation, and histogram for each set.
Then, you will roll two dice, adding together their sums, 10 and 1000 times.
Again, calculating the mean, standard deviation, and histogram for each set of rolls.

\section*{Requirements}
To fulfill this project, you must successfully program a script that utilizes several different functions and loops to simulate dice rolls and calculate the statistics.
Each function should have a practical name, be easy to read and understand, and function.
You should also comment appropriately throughout the program such that someone else could understand what is occurring.

\section*{Submission}
You will submit the following to the Canvas assignment box:
\begin{enumerate}
    \item A summary document with tables of the statistical parameters and copies of the generated histograms
    \item A well-documented code file OR
    \item An organized zipped archive of well-documented code source files. Make sure the main script is named properly.
\end{enumerate}

\marginnote[-0.75in]{Documents that contain the figures and printouts with the code, such as Jupyter notebooks, will also be accepted in lieu of the other submission items.}

Note that any submitted code needs to be able to be executed without any errors or importing any additional libraries.
Therefore, if you do anything tricky with libraries or external functions, be sure to include those source files appropriately.
Additionally, the main script should be named in the following format: \lstinline{h[#]_[ASSIGNMENT NAME]_[LAST NAME].[EXTENSION]}, for example, \lstinline{h3_dice_roll_duffy.ipynb}

\section*{Grading}
You will be graded on the following criteria:

\begin{margintable}
    \begin{tabular}{l | c}
        \toprule
        \multicolumn{1}{c|}{\textbf{Criterion}} & \textbf{Points} \\
        \midrule
        Efficacy & 60 \\
        Code Neatness & 20 \\
        Output Neatness & 20 \\
        Multilingual (EC) & 20 \\
        Additional Functions (EC) & 10 each \\
        \bottomrule
    \end{tabular}
\end{margintable}

\pagelayout{wide} % Remove margins

\section*{Extra Credit}
If you are willing to dig in a little bit more, this project has a couple of opportunities to earn extra credit points at the discretion of the instructor.
First, you may do this assignment in both Python and MATLAB. 
Being able to work with multiple programming languages is an extremely useful skill.
Python and MATLAB are also close enough in syntax that the code should not have to change much between them!

Additionally, you may write up to three different operations with the dice rolling and provide the same output for each.
For 10 points of extra credit each, you may write functions that subtract, multiply, and divide the rolls of two dice for 10 and 1000 rolls.
You will only be granted the extra credit points if the full output for each operation is provided!

\section*{Psuedocode}
For those that are new to programming, a suggested code layout is given below in psuedocode to guide you.

\begin{lstlisting}[style=kaolstplain, linewidth=1.5\textwidth]
    Program: Dice Roll
    
    Define 

    Function: Setup
        Initialize Serial communication for debugging
        Initialize input pins
        Initialize output pins
    
    Function: Loop
        Check for button pressed
        If button is pressed then
            If the current LED mode is the last LED mode
                Reset the current LED mode to the first one
            Else
                Increment the LED mode by 1
        Switch the current LED mode
            Case the current LED mode is set to 0
                Execute first LED function
            Case the current LED mode is set to 1
                Execute second LED function
            Case the current LED mode is set to 2
                Execute third LED function
            ...
    
    Function: First LED Function
        [YOUR CODE HERE]
    ...
    
    \end{lstlisting}

\end{document}