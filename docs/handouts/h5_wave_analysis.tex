% ===========================================
% HOMEWORK 5: Wave Analysis
% Written by: Braidan Duffy
%
% Date: 01/05/2023
% Last Revision: 01/05/2023
% ============================================

\documentclass[
	letterpaper, % Page size
	fontsize=10pt, % Base font size
	twoside=true, % Use different layouts for even and odd pages (in particular, if twoside=true, the margin column will be always on the outside)
	%open=any, % If twoside=true, uncomment this to force new chapters to start on any page, not only on right (odd) pages
	%chapterentrydots=true, % Uncomment to output dots from the chapter name to the page number in the table of contents
	numbers=noenddot, % Comment to output dots after chapter numbers; the most common values for this option are: enddot, noenddot and auto (see the KOMAScript documentation for an in-depth explanation)
]{kaobook}

\begin{document}

\chapter*{Homework 5: Wave Analysis}

\section*{Overview}
This homework assignment will get you more familiar with applied data science concepts by taking a look at a basic analysis of wave data!
For this assignment, you will plot the HOBO instrument time series captured from your deployments and plot the time series, similar to Homework 4.
However, you will now be performing a wave-by-wave analysis on those datasets as well as a normal sinusoidal wave.
Report the average wave height, $\bar{H}$, the significant wave height, $H_s$ or $H_{\frac{1}{3}}$, the root mean square wave height, $H_{rms}$, the average period, $\bar{T}$, and the significant wave period, $T_s$ or $T_{\frac{1}{3}}$.

You can do this either in Python or MATLAB, but keep in mind that you should use whatever programming language your group is using for the research topic. \marginnote{\emph{For extra credit:} You may do this assignment in both languages}

\textbf{\emph{GRADUATE STUDENTS:}} You will take this a step forward and perform a spectral analysis of the same wave data. 
Use the sine wave data as a litmus test for your code.
Compare and contrast the difference between the wave statistics between the spectral analysis and wave-by-wave analysis in a table by calculating the percent difference between them.
Discuss the differences adn why you think they are present. 

\marginnote{\emph{For extra credit: } Undergraduate students may perform the graduate section for 50 additional points!}

\section*{Requirements}
To fulfill this project, you must successfully program a script that utilizes several different functions and plots to display real time series data from our instruments!
Each function should have a practical name, be easy to read and understand, and work.
You should also comment appropriately throughout the program such that someone else could understand what is occurring.

You will need the following for a sine wave of 1-meter amplitude and 10-second period and each HOBO instrument dataset:

\begin{itemize}
    \item A function for each instrument that will import the time series data given a file path, start index, end index, and any additional arguments you need.
    \item A figure for each time series and plot the data with an appropriate title, labels, and axes scales.
    \item An additional figure that windows to a small number of waves and clearly shows the boundaries of your wave-by-wave analysis.
    \item A table with the average wave height, $\bar{H}$, the significant wave height, $H_s$ or $H_{\frac{1}{3}}$, the root mean square wave height, $H_{rms}$, the average period, $\bar{T}$, and the significant wave period, $T_s$ or $T_{\frac{1}{3}}$
    \item (\textbf{graduate students only}) A spectra figure 
    \item (\textbf{graduate students only}) A table containing the average $H_{m0}$, $T_{m01}$, $T_{m02}$, and peak period
    \item (\textbf{graduate students only}) A table summarizing the percent differences between the wave spectra and wave-by-wave analyses.
\end{itemize}

\marginnote[-0.5in]{\emph{Note:} All figures and tables need to have appropriate titles, labels, axes scales, and legends!}

\section*{Submission}
You will submit the following to the Canvas assignment box:
\begin{enumerate}
    \item A summary document with tables, figures, and any other relevant information from your analysis
    \item A well-documented code file OR
    \item An organized zipped archive of well-documented code source files. Make sure the main script is named properly.
\end{enumerate}

\marginnote{The main script should be named in the following format: \lstinline{h[#]_[ASSIGNMENT NAME]_[LAST NAME].[EXTENSION]}, for example, \lstinline{h5_wave_analysis_duffy.ipynb}}

\marginnote[-1.25in]{Documents that contain the figures and printouts with the code, such as Jupyter notebooks, will also be accepted in lieu of the other submission items.}

Note that any submitted code needs to be able to be executed without any errors. Any additional libraries need to be able to be installed using a package manager like pip.

% \pagelayout{wide} % Remove margins

\section*{Grading}
You will be graded on the following criteria:

\begin{table}[h!]
    \begin{tabular}{l | c}
        \toprule
        \multicolumn{1}{c|}{\textbf{Criterion}} & \textbf{Points} \\
        \midrule
        Efficacy & 60 \\
        Code Neatness & 20 \\
        Output Neatness & 20 \\
        Multilingual (EC) & 20 \\
        Above and Beyond (EC)\footnote{Undergraduate only} & 50 \\
        \bottomrule
    \end{tabular}
\end{table}

\section*{Extra Credit}
If you are willing to dig in a little bit more, this project has an opportunity to earn extra credit points at the discretion of the instructor.
You may do this assignment in both Python and MATLAB. 
Being able to work with multiple programming languages is an extremely useful skill.
Python and MATLAB are also close enough in syntax that the code should not have to change much between them.

Additionally, undergraduate students may elect to perform the graduate section of the assignment for 50\% of the assignments value.
In order to receive credit, the code must work perfectly according the given requirements.

\end{document}