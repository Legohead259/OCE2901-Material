% ===========================================
% HOMEWORK 4: Data Plotting
% Written by: Braidan Duffy
%
% Date: 01/05/2023
% Last Revision: 01/05/2023
% ============================================

\documentclass[
	letterpaper, % Page size
	fontsize=10pt, % Base font size
	twoside=true, % Use different layouts for even and odd pages (in particular, if twoside=true, the margin column will be always on the outside)
	%open=any, % If twoside=true, uncomment this to force new chapters to start on any page, not only on right (odd) pages
	%chapterentrydots=true, % Uncomment to output dots from the chapter name to the page number in the table of contents
	numbers=noenddot, % Comment to output dots after chapter numbers; the most common values for this option are: enddot, noenddot and auto (see the KOMAScript documentation for an in-depth explanation)
]{kaobook}

\begin{document}

\chapter*{Homework 4: Data Plotting}

\section*{Overview}
This homework assignment is intended to get you more familiar with the practical applications of data science with MATLAB and Python.
You will be writing an analysis script that can import a data file from the HOBO, Thetis, and Lowell instruments, parse said file, then plot the data with appropriate title, labels and axes scales. 
After plotting the raw data, you will be designing a digital filter and applying it to your data to "smooth" out the signals. 
\marginnote{\emph{For extra credit:} You may do this assignment in both languages}

\section*{Requirements}
To fulfill this project, you must successfully program a script that utilizes several different functions and plots to display real time series data from our instruments!
Each function should have a practical name, be easy to read and understand, and work.
You should also comment appropriately throughout the program such that someone else could understand what is occurring.

You will need a function for each instrument that will import the time series data given a file path, start index, end index, and any additional arguments you need.
Then, you will need to create a figure for each time series and plot the data with an appropriate title, labels, and axes scales.
You will then design a filter of your choice and apply it to the data to "smooth" out, or remove, any high frequency noise in the signal.
Explain the impact of your filter and why you chose your particular method.

\textbf{\emph{GRADUATE STUDENTS:}} You will take this a step forward by designing and applying a high pass, low pass, and band pass filter to each time series and explaining the impacts on your data.
Support your analysis with Bode plots that show the magnitude and phase shift of each filter.
In your submission, briefly discuss whether or not is it better to apply a band-pass filter or cascade a high-pass and low-pass filter together.
Also discuss the impact the filter phase shift will have on your time series and suggest programmatic ways to reduce this shift.
\marginnote{\emph{Hint:} MATLAB has already figured this part out, you just have to find out how and what they do.}

\pagebreak

\section*{Submission}
You will submit the following to the Canvas assignment box:
\begin{enumerate}
    \item A summary document with tables, figures, and any other relevant information from your analysis
    \item A well-documented code file OR
    \item An organized zipped archive of well-documented code source files. Make sure the main script is named properly.
\end{enumerate}

\marginnote{The main script should be named in the following format: \lstinline{h[#]_[ASSIGNMENT NAME]_[LAST NAME].[EXTENSION]}, for example, \lstinline{h4_data_plots_duffy.ipynb}}

\marginnote[-1.25in]{Documents that contain the figures and printouts with the code, such as Jupyter notebooks, will also be accepted in lieu of the other submission items.}

Note that any submitted code needs to be able to be executed without any errors. Any additional libraries need to be able to be installed using a package manager like pip.

% \pagelayout{wide} % Remove margins

\section*{Grading}
You will be graded on the following criteria:

\begin{table}[h!]
    \begin{tabular}{l | c}
        \toprule
        \multicolumn{1}{c|}{\textbf{Criterion}} & \textbf{Points} \\
        \midrule
        Summary Document & 60 \\
        Code Neatness & 20 \\
        Output Neatness & 20 \\
        Multilingual (EC) & 20 \\
        \bottomrule
    \end{tabular}
\end{table}

\section*{Extra Credit}
If you are willing to dig in a little bit more, this project has an opportunity to earn extra credit points at the discretion of the instructor.
You may do this assignment in both Python and MATLAB. 
Being able to work with multiple programming languages is an extremely useful skill.
Python and MATLAB are also close enough in syntax that the code should not have to change much between them.

\end{document}