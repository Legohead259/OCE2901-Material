% ===========================================
% Experiment Schedule
% Written by: Braidan Duffy
%
% Date: 01/05/2023
% Last Revision: 01/05/2023
% ============================================

\documentclass[
	letterpaper, % Page size
	fontsize=10pt, % Base font size
	twoside=true, % Use different layouts for even and odd pages (in particular, if twoside=true, the margin column will be always on the outside)
	%open=any, % If twoside=true, uncomment this to force new chapters to start on any page, not only on right (odd) pages
	%chapterentrydots=true, % Uncomment to output dots from the chapter name to the page number in the table of contents
	numbers=noenddot, % Comment to output dots after chapter numbers; the most common values for this option are: enddot, noenddot and auto (see the KOMAScript documentation for an in-depth explanation)
]{kaobook}

\begin{document}

\chapter*{Experiment Schedule}

\section*{Overview}
For this assignment, your group will be determining the primary and secondary testing days during the assigned testing week.
Give each calendar day a "surfability" score from 0-100, 100 being the most ideal day, 0 being the least. 
\marginnote{This includes weeks where you do not have to be testing.
Always prepare for the worst!}

You surfability score can factor in the tidal height, low tide time, predicted wave heights and periods, and the predicted wind speed and direction, with different weights for each.
For example, a day where low tide occurs around 0900, waves are expected to be 3-feet tall at 12 seconds, and winds are blowing offshore at 5-10 kts would receive a very high surfability score near 100.
For days where the opposite conditions occur, the surfability score would be very low.

Note that the instructor is required to be present while you are collecting field data.
Therefore, you should coordinate with them to determine their schedule and use it as a basis for your date selections.
Also, weekends are optional, but not encouraged.
The instructor may not be able to meet on those days and may not be able to guarantee they will be there either.

\marginnote{\emph{For extra credit:} Write a script or series of scripts that scrape data from online APIs and automatically calculate the surfability scores for your given range of days!}

\section*{Requirements}
Your calendar must include the following:

\begin{enumerate}
	\item Your team name at the top of the first page
	\item All of the calendar days between February 20 and April 21 with appropriate and easy-to-understand formatting - follow the layout of a typical calendar
	\item An assigned surfability score for each calendar day (excluding weekends, if you wish)
	\item Two highlighted days for every week in different colors. One color shall be designated primary, the other, secondary.
	\item A table on a separate page or document that details individual contributions to the calendar
\end{enumerate}

\pagebreak

\section*{Submission}
You will submit te following to the Canvas submission box:

\begin{enumerate}
	\item A calendar in PDF form that conforms to the Requirements section
	\item (optional) A separate document that details individual member's contributions to the calendar
	\item (extra credit) A script or zipped archive of scripts that calculate the surfability score automatically
\end{enumerate}

\marginnote[-0.5in]{An submitted code must follow the same submission rules as the previous homework assignments.}

\section*{Grading}
This assignment will use a pass-fail scheme.
Your calendar must present adequate detail for choosing the ideal dates for testing and clearly show this information.
If the instructor determines that the calendar does not meet the given requirements, the grade shall be a 0 for the whole team with the option to redo for half credit.

\section*{Extra Credit}
There is an opportunity for extra credit on this assignment, at the discretion of the instructor!
To receive this credit, you must write a script or series of scripts that scrape data from online APIs such as NOAA and Surfline to get predictions.
Then, your script must calculate your surfability score and report it for each potential testing day.
The code submission \emph{must} follow the submission guidelines for the previous homework assignments in order to receive credit.

The amount of credit granted by the instructor is up to their discretion but will not exceed 50\% of the overall assignment value.

\end{document}